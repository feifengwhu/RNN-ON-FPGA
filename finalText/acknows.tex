\chapter*{Agradecimentos}
\addcontentsline{toc}{chapter}{Acknowledgements}

Agradeço a todos os professores que se cruzaram comigo na minha vida e que, através da sua ação,
me transmitiram os conhecimentos necessários para me formar como Engenheiro e produzir este trabalho,
que espero que seja de uso para a comunidade científica e para a Humanidade. Uma palavra especial
ao meu orientador deste trabalho, o Prof. João Canas Ferreira, que desde o primeiro momento acolheu
esta proposta de trabalho e soube ver nela o seu potencial devido, e ao co-orientador, o Ivo Timóteo,
que mais do que minha referência para a parte teórica deste trabalho, é um bom amigo.

Agradeço todo o esforço da comunidade \textit{open-source} que desenvolveu grande parte das ferramentas
com que trabalhei, desde o \LaTeX~ em que escrevi este documento, ao \href{https://www.python.org/}{Python} e \href{http://www.numpy.org/}{Numpy}
usado na automatização de tarefas e no teste duma primeira versão software da rede. Agradeço também ao autor das \href{https://ece.uwaterloo.ca/~aplevich/Circuit_macros/}{Circuit Macros},
J.D. Aplevich, que muito foram úteis para a elaboração de quase todos os diagramas deste trabalho.

Uma última palavra de apreço para as pessoas especiais da minha vida. Os meus pais, que me formaram como Pessoa e me
incutiram força em todos os momentos da vida, do qual esta tese não é excepção, e à minha namorada, a Mafalda, que apesar
de não ser da minha área, sempre se entusiasmou e interessou por todos os progressos que fui atingindo ao longo destes meses.

\vspace{10mm}
\flushleft{José Pedro Castro Fonseca}
