% !TEX TS-program = pdflatex
% !TEX encoding = UTF-8 Unicode

% This is a simple template for a LaTeX document using the "article" class.
% See "book", "report", "letter" for other types of document.

\documentclass[11pt]{scrartcl} % use larger type; default would be 10pt

\usepackage[utf8]{inputenc} % set input encoding (not needed with XeLaTeX)

%%% Examples of Article customizations
% These packages are optional, depending whether you want the features they provide.
% See the LaTeX Companion or other references for full information.

%%% PAGE DIMENSIONS
\usepackage{geometry} % to change the page dimensions
\geometry{a4paper} % or letterpaper (US) or a5paper or....
% \geometry{margin=2in} % for example, change the margins to 2 inches all round
% \geometry{landscape} % set up the page for landscape
%   read geometry.pdf for detailed page layout information

\usepackage{graphicx} % support the \includegraphics command and options

% \usepackage[parfill]{parskip} % Activate to begin paragraphs with an empty line rather than an indent

%%% PACKAGES
\usepackage{titling}
\usepackage{booktabs} % for much better looking tables
\usepackage{array} % for better arrays (eg matrices) in maths
\usepackage{verbatim} % adds environment for commenting out blocks of text & for better verbatim
\usepackage{subfig} % make it possible to include more than one captioned figure/table in a single float
% These packages are all incorporated in the memoir class to one degree or another...

%%% HEADERS & FOOTERS
\usepackage{fancyhdr} % This should be set AFTER setting up the page geometry
\pagestyle{fancy} % options: empty , plain , fancy
\renewcommand{\headrulewidth}{0pt} % customise the layout...
\lhead{}\chead{}\rhead{}
\lfoot{}\cfoot{\thepage}\rfoot{}

%%% MATH PACKAGES
\usepackage{amsmath}

%%% END Article customizations

%%% The "real" document content comes below...

\title{Suggestions for the ``Dissertation preparation'' document.}
\author{Ivo Timóteo \\11$^{\text{th}}$ February, 2016}
\date{}
\pretitle{\begin{flushleft}\LARGE}
\posttitle{\end{flushleft}}
\preauthor{\begin{flushleft}}
\postauthor{\end{flushleft}\rule{\linewidth}{0.5mm}}
\predate{}
\postdate{}

\begin{document}
\maketitle

Next time make the .tex compile without errors so that I can suggest inline. 

Also, careful with commas and sentence structure. It's better to have multiple shorter, clear sentences than nested monsters requiring the reader to keep a stack of depth 4. And don't separate conjunctions with commas, please!

Furthermore, I am not correcting everything because I am too lazy to type. Make sure to give another very good passage over the text afterwards.

\subsection*{List of figures}
\begin{itemize}
	\item Use short captions on the figures otherwise LoF looks awful.
	\begin{verbatim}\caption[Short version for LoF]{Full caption}\end{verbatim}
\end{itemize}

\subsection*{Introduction}
\begin{itemize}
	\item {\bf REPLACE} [first sentence]\\{\bf BY} Brief overview of the report:
	\item {\bf REPLACE} the objectives of it\\{\bf BY} its objectives
	\item {\bf REPLACE} help me achieve it \\{\bf BY} help me achieve them
	\item {\bf SPELLING} theoreical
	\item {\bf REMOVE} for the unsuspicious reader that is not versed in that area of knowledge
	\item {\bf REMOVE} are presented, in consecutive order.
	\item {\bf REPLACE} I will use in the final solution, that will be outlined \\{\bf BY} will be used in the final solution, which is outlined
	\item {\bf REMOVE} exhaustively
	\item {\bf REPLACE} in tasks \\{\bf BY} as tasks
	\item {\bf REPLACE} Chart \\{\bf BY} chart
	\item {\bf REPLACE} utilizing during the course of my work \\{\bf BY} using in my work
	\item {\bf REPLACE} that I have already done to prepare this thesis \\{\bf BY} already concluded
\end{itemize}

\subsection*{Background}
\begin{itemize}
	\item {\bf REMOVE} The
	\item {\bf REPLACE} learning structures \\{\bf BY} models
	\item {\bf REMOVE} , itself,
	\item {\bf REPLACE} its \\{\bf BY} their
	\item {\bf SPELLING} beign
	\item {\bf REMOVE} that extend
	\item {\bf WTF} faltacitação
	\item {\bf COMMENT} Not really clear at all from the short explanation why LSTM improves over HMM and RNN which is what your sentence kind of makes us expect... 
\end{itemize}

\subsection*{Motivation}
\begin{itemize}
	\item {\bf REMOVE} [comma after realization]
	\item {\bf REMOVE} , hitherto,
	\item {\bf REPLACE} still needs to be \\{\bf BY} is
	\item {\bf REMOVE} possible
	\item {\bf REMOVE} [comma after automatic]
	\item {\bf REPLACE} , and saving \\{\bf BY} thus saving
	\item {\bf WTF} performance room
\end{itemize}

\subsection*{Objectives}
\begin{itemize}
	\item {\bf REPLACE} done \\{\bf BY} presented
	\item {\bf REMOVE} , to make justice to the name thesis, 
\end{itemize}

\subsection*{Theoretical background}
\begin{itemize}
	\item {\bf COMMENT} Not all learning is for classification; not all learning wants to extract features (e.g., we might just want a simple blackbox predictor); not all learning is supervised. Explicit vector representations have commas separating different components.
	\item {\bf REMOVE} diametrically
	\item {\bf COMMENT} refer that the sum of squares error is common because it's what appears under the assumption of Gaussian noise.
	\item {\bf COMMENT} ``try to extract higher level information from data''. Can you explain? Weren't you just looking for $y_i$? What is a higher level prediction?
\end{itemize}

\subsection*{Artificial Neural Networks}
\begin{itemize}
	\item {\bf COMMENT} ``try to mimic the Human Brain'' -- I don't think this is actually accepted at all; no one I know believes this to be true. Also, why the capitalization? Also, right after that you claim to know the underlying mathematical function of a human neuron. Finally, you present a linear model as being equivalent to a function which is incredibly reductive (can't you have a function that is a non-linear model of its inputs?).
	\item {\bf COMMENT} $b_0$ is usually called $w_0$ and you define $x = [1, x_1, \dots, x_n]$ giving you $y=f(\bf{w}^T\bf{x})$. Also, you have $N$ as number of data points and $n$ as number of features of the input which can be confusing.
	\item {\bf COMMENT} Step functions are usually between 0 and 1. On the plot it would be nicer if every function had the same range.
	\item {\bf COMMENT} Why are layers important?
	\item {\bf COMMENT} When discussing the equations (2.3) to (2.6) it would be nice to have the schematics of at least three chained neurons.
\end{itemize}

\subsection*{Recurrent Neural Networks}
\begin{itemize}
	\item {\bf REMOVE} , essentially,
	\item {\bf REPLACE} especially \\{\bf BY} commonly
	\item {\bf COMMENT} ``but no information about the correlation between themselves and the data points that preceded them did not influence the training step'' -- this is a double negative which I don't think you wanted.
	\item {\bf REPLACE} They were regarded as if no temporal relationship existed, and therefore each data point is conditionally independent of any other. \\{\bf BY} Temporal relationship is disregarded and  each data point considered conditionally independent of any other.
	\item {\bf REPLACE} truth \\{\bf BY} true
	\item {\bf REPLACE} , in which it is actually completely false, \\{\bf BY} such
	\item {\bf SPELLING} Simulatneous
	\item {\bf COMMENT} Stop the weird unnecessary capitalization of stuff :)
\end{itemize}

\subsection*{LSTMN}
\begin{itemize}
	\item {\bf REMOVE} basis
	\item {\bf REPLACE} its formulation \\{\bf BY} it
	\item {\bf REPLACE} structure [2] last year \\{\bf BY} model last year [2]
	\item {\bf SPELLING} presenation
	\item {\bf REPLACE} as well as the work \\{\bf BY} as well as of the work
	\item {\bf COMMENT} ``Hadamard elementwise matrix multiplication'' is {\bf very} redundant. Why is the activation function of the forget gate always sigmoid? Btw, even though sigmoid is usually seen as the logistic function it actually refers to the entire family of S-shaped functions.
	\item {\bf REMOVE} [commas around ``can be any'']
	\item {\bf REPLACE} As for the update rule is concerned, we have \\{\bf BY} The update rule is
	\item {\bf SPELLING} hipothetical
	\item {\bf COMMENT} how do you assess whether the behaviour of the weight is appropriate or not?
\end{itemize}

\subsection*{Proposed Solution}
\begin{itemize}
	\item {\bf REPLACE} work, their \\{\bf BY} work and their
	\item {\bf SPELLING} acheived 
\end{itemize}

\subsection*{State of the art}
\begin{itemize}
	\item {\bf COMMENT} ``bleeding edge'' is a bit too informal. Also, too many adjectives in 3.1; staggering, incredibly, astonishing, astounding, etc  
	\item {\bf SPELLING} Tthere
	\item {\bf SPELLING} restric
\end{itemize}

\subsection*{Work plan}
\begin{itemize}
	\item {\bf COMMENT} I don't know what to say about the first paragraph...
	\item {\bf SPELLING} subtaskswhich
	\item {\bf SPELLING} throgh
	\item {\bf SPELLING} asses
	\item {\bf SPELLING} [T3.1 is badly typed]
\end{itemize}



\end{document}
