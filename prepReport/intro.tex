\chapter{Introduction}\label{chap:intro}
Brief overview of the report:

In Chapter~\ref{chap:intro}, I will present a conceptual background to the problem at hand, exposing the background that surrounds it~\ref{sec:intro_back}, the motivation for the desired solution~\ref{sec:intro_mot}, its objectives~\ref{sec:intro_obj} and, lastly, the people that will help me achieve them~\ref{sec:intro_people}.

In Chapter~\ref{chap:theorBack}, the theoretical foundations will be layed out. Section~\ref{sec:tb_basicCon} presents the basic concepts of Machine Learning, and in the following sections, the theoretical details of ANNs, RNNs and LSTMs. Section~\ref{sec:training_lstm} provides a quick explanation of the training algorithm that will be used in the final solution, which will be outlined in Section~\ref{sec:propSol}.

In Chapter~\ref{chap:stateArt}, the state of the art of LSTMs, their applications~\ref{sec:sa_apps}, their hardware implementations~\ref{sec:sa_hardware} and the current work regarding the chosen training algorithm~\ref{sec:sa_training} are presented.

In Chapter~\ref{chap:workPlan}, the work plan for the Dissertation work is detailed as \emph{tasks} in Section~\ref{sec:wp_tasks_td}, and their temporal arrangement is pictured in a Gantt chart in Section~\ref{sec:wp_tasks_gantt}. Furthermore, the description of the Hardware and Software resources that I will be utilizing during the course of my work is done in Section~\ref{sec:wp_sh}

Finally, in Chapter~\ref{chap:earlyWork} I will present some of the work that I have already concluded.

\section{Background}\label{sec:intro_back}
Artificial Neural Networks are one of the most popular models in the field of Machine Learning. As the name suggests, their operation is inspired by the operation of the building blocks of our brain, the \textbf{neurons}. In spite of its high performance, one of their shortcomings is the fact that they cannot retain temporal dependences among incoming data samples, thus not being suitable to process time-series data, such as audio, video or other kinds of time-varying data streams, where current inputs have a high temporal dependence with previous and future inputs. 

To address this issue, several algorithms have been used, such as Hidden Markov Models (HMM) or Recurrent Neural Networks (RNN), but both of these methods fail to recall temporal dependences that extend over an large period of time, for the reasons that we will understand in Sections~\ref{sec:theorBack_rnn} and~\ref{sec:theorBack_lstm}.Thus, in 1997 Hochreiter et al. proposed a novel RNN structure~\cite{Hoch97}, the Long Short-Term Network (LSTM), where a memory cell was introduced, and the input/output/read/write access is controlled by individual gates that are activated both by the incoming data samples, but also by the outputs from the previous time-step (it is an RNN after all). They are one of the state of the art methods in Deep Learning nowadays, as we can attest in Section~\ref{sec:sa_apps}.

\section{Motivation}\label{sec:intro_mot}
Hitherto, and to the best of my knowledge, all of the published applications that use LSTM are software based, but the parallel nature of the structure hollers for a dedicated hardware realization that can dramatically increase its performance, something that  has only recently been done once~\cite{Chang15} (see Section~\ref{sec:sa_hardware} for further details), and although it improves the processing time when compared to a naïve software solution, it still lacks the ability to perform on-chip learning, and the learning process is performed offline, in a normal CPU, when it also could be speeded up in a dedicated hardware structure.

All these techniques are generally implemented in mainstream processors, making use of general high-level or low-level languages, where all the real parallelism is limited to the number of simultaneous threads that we can run on each physical core, which up to the date generally have between 2-8 cores (mobile devices and general use personal computers), 

In order to parallelize the computations to the fullest extent, a solution is to port it both to a Graphics Processing Unit (GPU) or even a Field-Programmable Gate Array (FPGA), but the portin process is not entierely automatic and to have the least performance drop possible, it has to be explicitly programmed in CUDA/OpenCL for GPUs, and a Hardware Description Language for FPGAs, with an increasing level of complexity and low-level details. This way, it is necessary to provide frameworks that can allow an FPGA to quickly reconfigure itself to run these kind of networks on demand, for a particular task that requires them, achieving a supposedly lower computation time, and unburdening the CPU from running it, thus saving performance for running other tasks related with the Operating System, for instance. Furthermore, when processing an incoming stream of highly dimensional data, or with high throughput, a CPU solution might not be scalable, and could benefit greatly from a dedicated hardware implementation.

\section{Objectives}\label{sec:intro_obj}
Taking into account the considerations presented in~\ref{sec:intro_mot}, I propose to develop a hardware implementation of an LSTM Network, \emph{with on-chip learning}, improving the performance, capability and flexibility of the existing solution of~\cite{Chang15}. Moreover, I also propose to benchmark this solution and try to compare it with the software performance results of~\cite{Tavear13} and, as a secondary objective, try to use my developed structure to replicate the some of the applications portrayed in~\ref{sec:sa_apps}. Lastly, I will work to make this structure reconfigurable on the go, and trying to minimize the reconfiguration time.

\section{People Involved}\label{sec:intro_people}
Besides me, the candidate to the Master's Degree, there are two more people involved, namely

\begin{itemize}
    \item \textbf{Supervisor} -- The \href{https://sigarra.up.pt/feup/pt/func\_geral.formview?p\_codigo=210963}{Professor João Canas Ferreira}, auxiliary professor at the Faculty of Engineering of the University of Porto.
    \item \textbf{Second Supervisor} -- \href{http://www.cl.cam.ac.uk/~ijpdmt2/}{Ivo Timóteo}, a Computer Science PhD candidate at Cambridge University, UK, in the field of Artificial Inteligence.
\end{itemize}

